\documentclass{beamer}
\usetheme{Berlin}
\usecolortheme{albatross}
\usepackage{harvard}

\title{What is language?}
\author{Liam Keeble}
\date{Partners}


\begin{document}

\begin{frame}
\maketitle
\end{frame}

\section{What is language?}

\begin{frame}
	\frametitle{What is language?}
	\begin{itemize}
		\item Some believe that language consists of only a single operation: Merge \cite{hauser2002faculty}.
		\item Merge is the ability to combine small, meaningful units in order to make larger, meaningful units.
		

	\end{itemize}

\end{frame}


\begin{frame}
	\frametitle{Merge}
	\begin{itemize}
		\item Others believe that language is more than just this \cite{jackendoff2005nature}.
	\item Combined units still need to be externalised, and humans still need to be able to process the relevance of language to the world around them.
	\item But both views believe that Merge forms a central operation of language.
	\end{itemize}

\end{frame}


\section{Breaking sentences}

\begin{frame}
	\frametitle{An analytic approach}
	\begin{itemize}
	\item Linguists have arrived at such debates via studying language from an analytic perspective.
	\item This means we try to understand language by breaking it down into its component parts and studying how they relate to other component parts.

	\end{itemize}

\end{frame}


\begin{frame}
	\frametitle{An example}
	\begin{itemize}
	\item Jane hugged John
	\item Now lets break it down.
	\item We have a noun, followed by a verb, followed by a noun.
	\item (John)(hugged)(Jane)
	\end{itemize}


\end{frame}

\begin{frame}
	\frametitle{An example}
	\begin{itemize}
	\item Notice we can replace each noun with a different noun, and the verb with certain different nouns.
	\item (The man)(hugged)(Jane)
	\item (The man)(hugged)(the woman)
	\item (The man)(found)(the woman)
	\end{itemize}
\end{frame}

\begin{frame}
	\frametitle{An example}
	\begin{itemize}
		\item Notice that (John) is interchangeable with (the man).
		\item And (Jane) is interchangeable with (the woman).
		\item This means that some parts of language are not just broken down into individual words, but phrases.
		\item This is one level of meaning.
		\item Individual phrases have meaning, and meanings are built upon meanings by combining smaller phrases, not words.
	\end{itemize}
\end{frame}

\section{Breaking words}

\begin{frame}
	\frametitle{Another example}

	\begin{itemize}
	\item We can break down words in a similar fashion.
	\item Words are made up of smaller units of meaning we call morphs.
	

	\end{itemize}
\end{frame}

\begin{frame}

	\frametitle{Another example}
	\begin{itemize}
	\item Take the word "Privatisation".
	\item The root of the word is "Private".
	\item this gives the word its core meaning.
	\item Adding the "-ise" morpheme to the word "Private" changes it from an adjective, to a verb.
	
	
	\end{itemize}
\end{frame}

\begin{frame}
	\frametitle{Another example}
	\begin{itemize}
	\item We can change the meaning again by adding the morpheme "-ation".
	\item The word then becomes "Privatisation", which is now a noun, instead of an adjective.
	\end{itemize}
\end{frame}


\section{Conclusions}

\begin{frame}
	\frametitle{Where to from here?}
	\begin{itemize}
	\item We have learnt that we can break down language at two levels: sentences into phrases, and words into morphemes.
	\item In the rest of the module we will look at these processes in more detail.
	\item However, instead of breaking language down as we have done here, we will be using the building blocks we have discussed into larger units of meaning.
	\end{itemize}

\end{frame}

\begin{frame}[allowframebreaks]
	\frametitle{References}
	\bibliographystyle{agsm}
	\bibliography{references}
\end{frame}

\end{document}
