\documentclass{beamer}
\usetheme{Berlin}
\usecolortheme{albatross}
\usepackage{harvard}

\title{What is language?}
\author{Liam Keeble}
\date{Partners}


\begin{document}

\begin{frame}
\maketitle
\end{frame}

\section{What is language?}

\begin{frame}
	\frametitle{What is language?}
	\begin{itemize}
		\item Some believe that language consists of only a single operation: Merge \cite{hauser2002faculty}.
		\item Merge is the ability to combine small, meaningful units in order to make larger, meaningful units.
		

	\end{itemize}

\end{frame}


\begin{frame}
	\frametitle{Merge}
	\begin{itemize}
		\item Others believe that language is more than just this \cite{jackendoff2005nature}.
	\item Combined units still need to be externalised, and humans still need to be able to process the relevance of language to the world around them.
	\item But both views believe that Merge forms a central operation of language.
	\end{itemize}

\end{frame}


\section{Breaking sentences into phrases}

\begin{frame}
	\frametitle{An analytic approach}
	\begin{itemize}
	\item Linguists have arrived at such debates via studying language from an analytic perspective.
	\item This means we try to understand language by breaking it down into its component parts and studying how they relate to other component parts.

	\end{itemize}

\end{frame}


\begin{frame}
	\frametitle{An example}
	\begin{itemize}
	\item Jane hugged John
	\item Now lets break it down.
	\item We have a noun, followed by a verb, followed by a noun.
	\item (John)(hugged)(Jane)
	\end{itemize}


\end{frame}

\begin{frame}
	\frametitle{An example}
	\begin{itemize}
	\item Notice we can replace each noun with a different noun, and the verb with certain different nouns.
	\item (The man)(hugged)(Jane)
	\item (The man)(hugged)(the woman)
	\item (The man)(found)(the woman)
	\end{itemize}
\end{frame}

\begin{frame}
	\frametitle{An example}
	\begin{itemize}
		\item Notice that (John) is interchangeable with (the man).
		\item And (Jane) is interchangeable with (the woman).
		\item This means that some parts of language are not just broken down into individual words, but phrases.
		\item This is one level of meaning.
		\item Individual phrases have meaning, and meanings are built upon meanings by combining smaller phrases, not words.
	\end{itemize}
\end{frame}

\section{Breaking phrases into words}


\section{Breaking words into morphs and sounds}


\begin{frame}[allowframebreaks]
	\frametitle{References}
	\bibliographystyle{agsm}
	\bibliography{references}
\end{frame}

\end{document}
