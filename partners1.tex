
\documentclass{beamer}

\usetheme{Berlin}
\title{Structure and Meaning 1}

\subtitle{Introduction}
\author{Liam Keeble}
\institute{School of English Literature, Language and Linguistics}
\date{}

\begin{document}

\section{Introduction to me}

\frame{\titlepage}



\begin{frame}
\frametitle{Introduction to me}

\begin{itemize}
\item Name: Liam Keeble
\item Email: L.Keeble1@newcastle.ac.uk
\item Field of study: Meta-science of comparative cognition and sociolinguistics
\item Background: From linguistics to animal communication to comparative cognition

\end{itemize}

\end{frame}


\section{Introduction to the module}

\begin{frame}
\frametitle{Module objectives}
\begin{itemize}
\item To give you the tools to deconstruct/construct language from its most basic building blocks
\item To understand where meaning enters linguistic structures
\item To find patterns in these structures

\end{itemize}

\end{frame}

\begin{frame}
\frametitle{Module structure}

\begin{itemize}
\item Break language down into basic blocks of meaning
\item Build words from their meaningful units
\item Build phrases from words to create even more meaningful units
\item Move phrases around to alter meaning further
\end{itemize}

\end{frame}

\section{Introduction to our approach}

\begin{frame}
\frametitle{Breaking language down}
\begin{itemize}
\item Notice that our phrases and sentences consist of multiple individual phrases with their own meaning.
\item John shouted at the cat on the roof
\item {[[John shouted] at [[the cat] on [the roof]]]}
\end{itemize}
\end{frame}

\begin{frame}
\frametitle{Breaking language down}
\begin{itemize}
\item We can also break our words down into individual units of meaning
\item The word `shouted' can be broken down into `shout' and `-ed'
\item `shout' has meaning on its own, `-ed' does not, but gives another word new meaning
\end{itemize}
\end{frame}


\begin{frame}
\frametitle{Finding patterns}
\begin{itemize}
\item The ways in which words and phrases can be combined follow rules
\item We want to try to identify these rules across the English language

\end{itemize}

\end{frame}

\section{Why?}
\begin{frame}
\frametitle{Why?}
\begin{itemize}
\item Language exists in the brain
\item It is unique to humans
\item But differs in specific ways between speakers of different languages
\item What are the similarities and differences between huan languages?
\item And what does this tell us about the human mind?

\end{itemize}

\end{frame}

\begin{frame}
	\frametitle{Next up...}
	\begin{itemize}
	\item Next we will look at what meaning is,
	\item How we understand it,
	\item Its most basic building blocks,
	\item And how we begin to build language from them

	\end{itemize}
\end{frame}


\end{document}
